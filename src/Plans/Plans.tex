\documentclass{article}
\usepackage[utf8]{inputenc}
\usepackage[paper=a4paper]{geometry}
\usepackage{amsmath}
%\usepackage{tikz}

\title{Audacious Plans}
\author{Rahul Manavalan}
\date{}

\begin{document}

\maketitle

This file is a composite collection of ideas, some vetted, some freehand, some realistic, some far-fetched. 
Nonetheless, these are my ideas and my own.
It ought to be said, the ideas could also be inspired by some of the discussions with my peers.
In such a case, due credit will be referenced with the permisison of the involved party.


\section{Date: 24/08/2022}
It is my intention to develop a novel model reduction algorithm for a class of partial differential equations.
This endeavour is quite challenging as this involves a number of difficulties, namely
\begin{enumerate}
    \item The volume of available algorithms is huge.
    \item Implementations of these algorithms is scarce.
    \item Comparison of any kind, of the different algorithms is impossible.
    \item Problems to which these algorithms are applied to, are different.
\end{enumerate}

In order to truly introduce a novel algorithm extensive study must be carried out and these studies need to be backed up,
by implementing the algorithms for a wide variety of problems. The problem here is the availablity of appropriate opensource solvers,
to the classes of PDEs that we want to solve. (As we shall consider the efficacy of intrusive and non-intrusive algorithms together.)

As a result classes of high dimensional, computationally challenging and frequently occuring PDEs in Science and Engineering need to be chosen,
test cases properly studied and efficient numerical solvers need to be implemented using the best possible method for that problem. Needless to say,
we should try to solve all the problems using the concept of Method of Lines as this allows for easy transformation in the latent space, especially when dealing with
projection based Model Reduction Methods. 

The question of best solver is contentious. In my vocabulary, any solver that allows for a fast implementation with sufficient accuray qualifies as the best.
It should also be noted that in some instances, one might need to resort to using approximation techniques such as sparse grids to ensure that a surrogate to the model can be developed.
This is especially handy should one need to generate data for some of the Operator Regression paradigm of surrogates.

Digressions aside, there are three areas of my personal interests that I would like to develop solvers for, namely
\begin{enumerate}
    \item Fluid Structure Interaction of Turbulent Flow past a Hypersonic Plane.
    \item Chemical Vapour Deposition of a Silicon Film on a Substrate.
    \item Non-linear Diffusion of Species in a Lithium Ion Cell.
\end{enumerate}

The level of detail to which the solvers need to be developed for these applications is yet to be deteremined.
In any case, it would make sense to test the results of my simulation with actual experimental data, so it might well be that these problems could be supplanted by ones where real data is available.

In the time when the solvers are being prepared, model reduction algorithms need to be reviewed, studied and implemented.
They need to be tested with some of the prototypical examples from the Computational Reality Book that are solved using the Spectral Methods
Thus before one can start working on algorithms one needs to have a good understanding of the spectral methods and using them to define the forcing function to some of the toy-practical problems out there.

For the algorithms in Model Reduction, Felix's references and Benner's textbooks should cover adequate ground. We shall adopt the algorithms in these resources first before venturing onto other modern, fancy techniques.
That said, we shall place focus on the following disciplines of Model Reduction.
\begin{enumerate}
    \item Operator Inference
    \item Projection Based Methods
    \item Reduced Basis Method
\end{enumerate}

\end{document}
