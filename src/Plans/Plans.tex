\documentclass{article}
\usepackage[utf8]{inputenc}
\usepackage[paper=a4paper]{geometry}
\usepackage{amsmath}
\usepackage{tikz}

\title{Audacious Plans}
\author{Rahul Manavalan}
\date{}

\begin{document}

\maketitle

This file is a composite collection of ideas, some vetted, some freehand, some realistic, some far-fetched. 
Nonetheless, these are my ideas and my own.
It ought to be said, the ideas could also be inspired by some of the discussions with my peers.
In such a case, due credit will be referenced with the permisison of the involved party.


\section{Date: 24/08/2022}
It is my intention to develop a novel model reduction algorithm for a class of partial differential equations.
This endeavour is quite challenging as this involves a number of difficulties, namely
\begin{enumerate}
    \item The volume of available algorithms is huge.
    \item Implementations of these algorithms is scarce.
    \item Comparison of any kind, of the different algorithms is impossible.
    \item Problems to which these algorithms are applied to, are different.
\end{enumerate}

In order to truly introduce a novel algorithm extensive study must be carried out and these studies need to be backed up,
by implementing the algorithms for a wide variety of problems. The problem here is the availablity of appropriate opensource solvers,
to the classes of PDEs that we want to solve. (As we shall consider the efficacy of intrusive and non-intrusive algorithms together.)

As a result classes of high dimnensional, computationally challenging and frequently occuring PDEs in Science and Engineering need to be chosen,
test cases properly studied and efficient numerical solvers need to be implemented using the best possible method for that problem. Needless to say,
we should try to solve all the problems using the concept of Method of Lines as this allows for easy transformation in the latent space, especially when dealing with
projection based Model Reduction Methods. 

The question of best solver is contentious. In my vocabulary, any solver that allows for a fast implementation with sufficient accuray qualifies as the best.
It should also be noted that in some instances, one might need to resort to using approximation techniques such as sparse grids to ensure that a surrogate to the model can be developed.
This is especially handy should one need to generate data for some of the Operator Regression paradigm of surrogates.



\end{document}