\subsection{Symmetries}

Consider a vector space $X$ endowed with a metric $m$ , i.e. a Metric Space $M(X,m)$. \\
The problem of determining the geometric symmetries of the vector space $X$ corresponds to
finding invariants of $X$ under an arbitrary transformation $T$. \\

One immediately has to answer the following questions.
\begin{enumerate}
    \item What is the vector space $X$?
    \item What is the metric $m$ that one is interested in?
    \item What are the invertible transformations $T$ defined on $X$?
\end{enumerate}

\subsubsection{Case 1}
$X \subset \mathbb{R}$ defines the most trivial space. 
Principally, a point on $X$ can stay where it was or be shifted by a finite or a non-finite value.
The former corresponds to an Identity Transformation $I$ while the latter corresponds to a Translation $T$.
A useful metric that $X$ could be equipped with is the Manhattan distance $d_{m}$. In addition, $d_{m}$ in invariant under $T$.


\subsubsection{Case 2}
